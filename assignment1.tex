%----------------------------------------------------------------------------------------
%	PACKAGES AND OTHER DOCUMENT CONFIGURATIONS
%----------------------------------------------------------------------------------------

\documentclass{article}

\input{structure.tex} % Include the file specifying the document structure and custom commands

%----------------------------------------------------------------------------------------
%	ASSIGNMENT INFORMATION
%----------------------------------------------------------------------------------------

\title{Cryptography, ITC8240 Assignment \#1} % Title of the assignment

\author{Oskar Pihlak} % Author name and email address

\date{TalTech --- \today} % University, school and/or department name(s) and a date

%----------------------------------------------------------------------------------------

\begin{document}

\maketitle % Print the title

%----------------------------------------------------------------------------------------
%	INTRODUCTION
%----------------------------------------------------------------------------------------

\section*{Introduction} % Unnumbered section

This is the Assignment \#1 submission for the Cryptography course, written in LaTeX.\\
It's assumed that the shift cipher is defined as \textit{$C_{alpha}$} = ABCDEFGHIJKLMNOPQRSTUVWXYZ


%----------------------------------------------------------------------------------------
%	PROBLEM 1
%----------------------------------------------------------------------------------------

\section{Task 1: Ciphertext evaluation} % Numbered section

Starting with plaintext \textit{$T_{plain}$} = \textbf{BLOCKCHAIN}. \\

$S_1$ is a shift cipher with key $k_{S_1}$ = 9. \\
All the letters in \textit{$T_{plain}$} will be shifted by 9 characters in the alphabet.\\
Resulting in \textit{$T_{S_1}$} = \textbf{KUXLTLQJRW}. \\
 
$P_1$ is a permutation cipher with a key $k_{P_1}$ = (5, 1, 3, 2, 4) \\
Since $k_{P_1}$ length is 5, \textit{$T_{S_1}$} will be splitted into two 5 letter chunks\\
\begin{math}
    \mathit{{T_{S_1 chunks}} = [KUXLT, LQJRW]}
\end{math}\\
We apply the permutation cipher to each of the chunks and combine them together.\\
\textit{$T_{S_1 P_1}$} =  \textbf{TKXULWLJQR}
\\

$S_2$ is a shift cipher with key $k_{S_2}$ = 19. \\
All the letters in \textit{$T_{S_1 P_1}$} will be shifted by 19 characters in the alphabet.\\
Resulting in \textit{$T_{S_1 P_1 S_2}$} = \textbf{MDQNE PECJK}. \\

$P_2$ is a permutation cipher with a key $k_{P_2}$ = (3, 1, 4, 2, 5) \\
Since $k_{P_2}$ length is 5, \textit{$T_{S_1 P_1 S_2}$} will be splitted into two 5 letter chunks\\
\begin{math}
    \mathit{{T_{S_1 P_1 S_2 chunks}} = [MDQNE, PECJK]}
\end{math}\\
We apply the permutation cipher to each of the chunks and combine them together.\\
\textit{$T_{S_1 P_1 S_2 P_2}$} =  \textbf{QMNDE CPJEK}
\\
\\
Answer: The Ciphertext is \textbf{QMNDECPJEK}

%----------------------------------------------------------------------------------------
%	PROBLEM 2
%----------------------------------------------------------------------------------------

\section{Task 2}
Starting with plaintext \textit{$T_{plain}$} = \textbf{FRIENDSMAKETHEWORSTENEMIES}. \\

1. Encrypt the plaintext using Vigenere cipher with the key \textit{k} = \textbf{LIST} \\
The key k will repeat across the the entirety of \textit{$T_{plain}$} (24 characters) resulting in \\
the \textbf{6\textit{k}} keystream \textit{ks} = LISTLISTLISTLISTLISTLIST; \\ 
We will use a matrix to map the \textit{$T_{plain}$} into the cipher values.\\ 
Resulting in \textit{$T_{vignere}$} = QZAXYLKFLSWMSMOHCALXYMEBPA \\\\
$
\begin{array}{l|cccc}
    & A    & B    & \dots  & Z    \\ \hline
A   & A    & C & \dots  & Z \\
B   & B    & D & \dots  & A \\
\vdots & \vdots & \vdots & \ddots & \vdots \\
Z   & Z    & A & \dots  & Y
\end{array}
$\\ \\


2. Calculate the index of coincidence of the plaintext.\\
The index of coincidence can be explained with the given formula.
\[ IC = \sum_{i=A}^{i=z} \frac{n_i(n_i -1)}{N(N - 1)} \]
All alphabet letters are looped over. \\
\textit{$n_i$} is the current letter that is being looped over.\\
\textit{N} is the total number of letters in the given text\\\\
IC(\textit{$T_{plain}$}) = \textbf{0.07077}
\\

3. Calculate the index of coincidence of the ciphertext.\\
IC(\textit{$T_{vignere}$}) = \textbf{0.03692}

%----------------------------------------------------------------------------------------

%----------------------------------------------------------------------------------------
%	PROBLEM 3
%----------------------------------------------------------------------------------------

\section{Task 3}
Starting with plaintext \textit{$T_{plain}$} = \textbf{SURFACE} and ciphertext \textit{$T_{ciphered}$} = \textbf{NJCAXTP}. \\
We know that an affine cipher was used.\\

1. What is the encryption key?\\
Some of the encryption pairs are (11, 23), (37 23) and (63, 23)\\


2. What is the decryption key?\\
Some of the encryption pairs are (7, 23), (45 23) and (71, 23)\\



%----------------------------------------------------------------------------------------


\section{Task 4}
Starting with plaintext \textit{$m_1$} = \textbf{DOUGH} and plaintext \textit{$m_2$} = \textbf{GLORY}. \\
Message \textit{$m_1$} encryption with \textit{k} resulted in a binary ciphertext \textit{$c_1$} = 1000000110001010001000100.
Extracting the cipher from \textit{$c_1$} via $\oplus$ operations gives us \textit{$cipher$} = 1001101000100010010000011.
Using \textit{$cipher$} on \textit{$m_2$} we get the encryption \textit{$c_2$} = 1010100011111111010111011.

%----------------------------------------------------------------------------------------


\section{Task 5}
We have an encryption scheme \textbf{\textit{$c = m \wedge~k$}} to evaluate for perfect secrecy.\\
A cryptosystem has perfect secrecy if for any message x and any encipherment y, p(x|y)=p(x). \\
An encryption scheme is perfectly secret if the ciphertext distribution is independent from the message,\\meaning
every message induces the same ciphertext distribution. \\
The condition declared in \textit{c} seems to have the message dependent in the encryption meaning the scheme is not secure.

%----------------------------------------------------------------------------------------

\end{document}
