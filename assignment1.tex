%----------------------------------------------------------------------------------------
%	PACKAGES AND OTHER DOCUMENT CONFIGURATIONS
%----------------------------------------------------------------------------------------

\documentclass{article}

%----------------------------------------------------------------------------------------
%	PACKAGES AND OTHER DOCUMENT CONFIGURATIONS
%----------------------------------------------------------------------------------------

\usepackage{amsmath,amsfonts,stmaryrd,amssymb} % Math packages

\usepackage{enumerate} % Custom item numbers for enumerations

\usepackage[ruled]{algorithm2e} % Algorithms

\usepackage[framemethod=tikz]{mdframed} % Allows defining custom boxed/framed environments

\usepackage{listings} % File listings, with syntax highlighting
\lstset{
	basicstyle=\ttfamily, % Typeset listings in monospace font
}

%----------------------------------------------------------------------------------------
%	DOCUMENT MARGINS
%----------------------------------------------------------------------------------------

\usepackage{geometry} % Required for adjusting page dimensions and margins

\geometry{
	paper=a4paper, % Paper size, change to letterpaper for US letter size
	top=2.5cm, % Top margin
	bottom=3cm, % Bottom margin
	left=2.5cm, % Left margin
	right=2.5cm, % Right margin
	headheight=14pt, % Header height
	footskip=1.5cm, % Space from the bottom margin to the baseline of the footer
	headsep=1.2cm, % Space from the top margin to the baseline of the header
	%showframe, % Uncomment to show how the type block is set on the page
}

%----------------------------------------------------------------------------------------
%	FONTS
%----------------------------------------------------------------------------------------

\usepackage[utf8]{inputenc} % Required for inputting international characters
\usepackage[T1]{fontenc} % Output font encoding for international characters

\usepackage{newtxtext} % Use the XCharter fonts

%----------------------------------------------------------------------------------------
%	COMMAND LINE ENVIRONMENT
%----------------------------------------------------------------------------------------

% Usage:
% \begin{commandline}
%	\begin{verbatim}
%		$ ls
%		
%		Applications	Desktop	...
%	\end{verbatim}
% \end{commandline}

\mdfdefinestyle{commandline}{
	leftmargin=10pt,
	rightmargin=10pt,
	innerleftmargin=15pt,
	middlelinecolor=black!50!white,
	middlelinewidth=2pt,
	frametitlerule=false,
	backgroundcolor=black!5!white,
	frametitle={Command Line},
	frametitlefont={\normalfont\sffamily\color{white}\hspace{-1em}},
	frametitlebackgroundcolor=black!50!white,
	nobreak,
}

% Define a custom environment for command-line snapshots
\newenvironment{commandline}{
	\medskip
	\begin{mdframed}[style=commandline]
}{
	\end{mdframed}
	\medskip
}

%----------------------------------------------------------------------------------------
%	FILE CONTENTS ENVIRONMENT
%----------------------------------------------------------------------------------------

% Usage:
% \begin{file}[optional filename, defaults to "File"]
%	File contents, for example, with a listings environment
% \end{file}

\mdfdefinestyle{file}{
	innertopmargin=1.6\baselineskip,
	innerbottommargin=0.8\baselineskip,
	topline=false, bottomline=false,
	leftline=false, rightline=false,
	leftmargin=2cm,
	rightmargin=2cm,
	singleextra={%
		\draw[fill=black!10!white](P)++(0,-1.2em)rectangle(P-|O);
		\node[anchor=north west]
		at(P-|O){\ttfamily\mdfilename};
		%
		\def\l{3em}
		\draw(O-|P)++(-\l,0)--++(\l,\l)--(P)--(P-|O)--(O)--cycle;
		\draw(O-|P)++(-\l,0)--++(0,\l)--++(\l,0);
	},
	nobreak,
}

% Define a custom environment for file contents
\newenvironment{file}[1][File]{ % Set the default filename to "File"
	\medskip
	\newcommand{\mdfilename}{#1}
	\begin{mdframed}[style=file]
}{
	\end{mdframed}
	\medskip
}

%----------------------------------------------------------------------------------------
%	NUMBERED QUESTIONS ENVIRONMENT
%----------------------------------------------------------------------------------------

% Usage:
% \begin{question}[optional title]
%	Question contents
% \end{question}

\mdfdefinestyle{question}{
	innertopmargin=1.2\baselineskip,
	innerbottommargin=0.8\baselineskip,
	roundcorner=5pt,
	nobreak,
	singleextra={%
		\draw(P-|O)node[xshift=1em,anchor=west,fill=white,draw,rounded corners=5pt]{%
		Question \theQuestion\questionTitle};
	},
}

\newcounter{Question} % Stores the current question number that gets iterated with each new question

% Define a custom environment for numbered questions
\newenvironment{question}[1][\unskip]{
	\bigskip
	\stepcounter{Question}
	\newcommand{\questionTitle}{~#1}
	\begin{mdframed}[style=question]
}{
	\end{mdframed}
	\medskip
}

%----------------------------------------------------------------------------------------
%	WARNING TEXT ENVIRONMENT
%----------------------------------------------------------------------------------------

% Usage:
% \begin{warn}[optional title, defaults to "Warning:"]
%	Contents
% \end{warn}

\mdfdefinestyle{warning}{
	topline=false, bottomline=false,
	leftline=false, rightline=false,
	nobreak,
	singleextra={%
		\draw(P-|O)++(-0.5em,0)node(tmp1){};
		\draw(P-|O)++(0.5em,0)node(tmp2){};
		\fill[black,rotate around={45:(P-|O)}](tmp1)rectangle(tmp2);
		\node at(P-|O){\color{white}\scriptsize\bf !};
		\draw[very thick](P-|O)++(0,-1em)--(O);%--(O-|P);
	}
}

% Define a custom environment for warning text
\newenvironment{warn}[1][Warning:]{ % Set the default warning to "Warning:"
	\medskip
	\begin{mdframed}[style=warning]
		\noindent{\textbf{#1}}
}{
	\end{mdframed}
}

%----------------------------------------------------------------------------------------
%	INFORMATION ENVIRONMENT
%----------------------------------------------------------------------------------------

% Usage:
% \begin{info}[optional title, defaults to "Info:"]
% 	contents
% 	\end{info}

\mdfdefinestyle{info}{%
	topline=false, bottomline=false,
	leftline=false, rightline=false,
	nobreak,
	singleextra={%
		\fill[black](P-|O)circle[radius=0.4em];
		\node at(P-|O){\color{white}\scriptsize\bf i};
		\draw[very thick](P-|O)++(0,-0.8em)--(O);%--(O-|P);
	}
}

% Define a custom environment for information
\newenvironment{info}[1][Info:]{ % Set the default title to "Info:"
	\medskip
	\begin{mdframed}[style=info]
		\noindent{\textbf{#1}}
}{
	\end{mdframed}
}
 % Include the file specifying the document structure and custom commands

%----------------------------------------------------------------------------------------
%	ASSIGNMENT INFORMATION
%----------------------------------------------------------------------------------------

\title{Cryptography, ITC8240 Assignment \#1} % Title of the assignment

\author{Oskar Pihlak} % Author name and email address

\date{TalTech --- \today} % University, school and/or department name(s) and a date

%----------------------------------------------------------------------------------------

\begin{document}

\maketitle % Print the title

%----------------------------------------------------------------------------------------
%	INTRODUCTION
%----------------------------------------------------------------------------------------

\section*{Introduction} % Unnumbered section

This is the Assignment \#1 submission for the Cryptography course, written in LaTeX.\\
It's assumed that the shift cipher is defined as \textit{$C_{alpha}$} = ABCDEFGHIJKLMNOPQRSTUVWXYZ


%----------------------------------------------------------------------------------------
%	PROBLEM 1
%----------------------------------------------------------------------------------------

\section{Task 1: Ciphertext evaluation} % Numbered section

Starting with plaintext \textit{$T_{plain}$} = \textbf{BLOCKCHAIN}. \\

$S_1$ is a shift cipher with key $k_{S_1}$ = 9. \\
All the letters in \textit{$T_{plain}$} will be shifted by 9 characters in the alphabet.\\
Resulting in \textit{$T_{S_1}$} = \textbf{KUXLTLQJRW}. \\
 
$P_1$ is a permutation cipher with a key $k_{P_1}$ = (5, 1, 3, 2, 4) \\
Since $k_{P_1}$ length is 5, \textit{$T_{S_1}$} will be splitted into two 5 letter chunks\\
\begin{math}
    \mathit{{T_{S_1 chunks}} = [KUXLT, LQJRW]}
\end{math}\\
We apply the permutation cipher to each of the chunks and combine them together.\\
\textit{$T_{S_1 P_1}$} =  \textbf{TKXULWLJQR}
\\

$S_2$ is a shift cipher with key $k_{S_2}$ = 19. \\
All the letters in \textit{$T_{S_1 P_1}$} will be shifted by 19 characters in the alphabet.\\
Resulting in \textit{$T_{S_1 P_1 S_2}$} = \textbf{MDQNE PECJK}. \\

$P_2$ is a permutation cipher with a key $k_{P_2}$ = (3, 1, 4, 2, 5) \\
Since $k_{P_2}$ length is 5, \textit{$T_{S_1 P_1 S_2}$} will be splitted into two 5 letter chunks\\
\begin{math}
    \mathit{{T_{S_1 P_1 S_2 chunks}} = [MDQNE, PECJK]}
\end{math}\\
We apply the permutation cipher to each of the chunks and combine them together.\\
\textit{$T_{S_1 P_1 S_2 P_2}$} =  \textbf{QMNDE CPJEK}
\\
\\
Answer: The Ciphertext is \textbf{QMNDECPJEK}

%----------------------------------------------------------------------------------------
%	PROBLEM 2
%----------------------------------------------------------------------------------------

\section{Task 2}
Starting with plaintext \textit{$T_{plain}$} = \textbf{FRIENDSMAKETHEWORSTENEMIES}. \\

1. Encrypt the plaintext using Vigenere cipher with the key \textit{k} = \textbf{LIST} \\
The key k will repeat across the the entirety of \textit{$T_{plain}$} (24 characters) resulting in \\
the \textbf{6\textit{k}} keystream \textit{ks} = LISTLISTLISTLISTLISTLIST; \\ 
We will use a matrix to map the \textit{$T_{plain}$} into the cipher values.\\ 
Resulting in \textit{$T_{vignere}$} = QZAXYLKFLSWMSMOHCALXYMEBPA \\\\
$
\begin{array}{l|cccc}
    & A    & B    & \dots  & Z    \\ \hline
A   & A    & C & \dots  & Z \\
B   & B    & D & \dots  & A \\
\vdots & \vdots & \vdots & \ddots & \vdots \\
Z   & Z    & A & \dots  & Y
\end{array}
$\\ \\


2. Calculate the index of coincidence of the plaintext.\\
The index of coincidence can be explained with the given formula.
\[ IC = \sum_{i=A}^{i=z} \frac{n_i(n_i -1)}{N(N - 1)} \]
All alphabet letters are looped over. \\
\textit{$n_i$} is the current letter that is being looped over.\\
\textit{N} is the total number of letters in the given text\\\\
IC(\textit{$T_{plain}$}) = \textbf{0.07077}
\\

3. Calculate the index of coincidence of the ciphertext.\\
IC(\textit{$T_{vignere}$}) = \textbf{0.03692}

%----------------------------------------------------------------------------------------

%----------------------------------------------------------------------------------------
%	PROBLEM 3
%----------------------------------------------------------------------------------------

\section{Task 3}
Starting with plaintext \textit{$T_{plain}$} = \textbf{SURFACE} and ciphertext \textit{$T_{ciphered}$} = \textbf{NJCAXTP}. \\
We know that an affine cipher was used.\\

1. What is the encryption key?\\
Some of the encryption pairs are (11, 23), (37 23) and (63, 23)\\


2. What is the decryption key?\\
Some of the encryption pairs are (7, 23), (45 23) and (71, 23)\\



%----------------------------------------------------------------------------------------


\section{Task 4}
Starting with plaintext \textit{$T_{plain}$} = \textbf{SURFACE} and ciphertext \textit{$T_{ciphered}$} = \textbf{NJCAXTP}. \\




%----------------------------------------------------------------------------------------



\end{document}
