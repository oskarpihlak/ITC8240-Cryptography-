%----------------------------------------------------------------------------------------
%	PACKAGES AND OTHER DOCUMENT CONFIGURATIONS
%----------------------------------------------------------------------------------------

\documentclass{article}
\usepackage{mathtools}
\usepackage{float}

\input{structure.tex} % Include the file specifying the document structure and custom commands

%----------------------------------------------------------------------------------------
%	ASSIGNMENT INFORMATION
%----------------------------------------------------------------------------------------

\title{Cryptography, ITC8240 Assignment \#3} % Title of the assignment

\author{Oskar Pihlak} % Author name and email address

\date{TalTech --- \today} % University, school and/or department name(s) and a date

%----------------------------------------------------------------------------------------

\begin{document}

\maketitle % Print the title

%----------------------------------------------------------------------------------------
%	INTRODUCTION
%----------------------------------------------------------------------------------------

\section*{Introduction} % Unnumbered section

This is the Assignment \#3 submission for the Cryptography course, written in LaTeX.\\



%----------------------------------------------------------------------------------------
%	2.1 TASK 1
%----------------------------------------------------------------------------------------

\section*{Task 1 (12 points) - Groups and Number theory} % Numbered section
\begin{itemize}
    \item Is ($Q$, $+$) (Rationals under addition) a group? Motivate your answer.
\end{itemize}
asdasdasdasd

\begin{itemize}
    \item Is ($Z_{33}$ \textbackslash {0}, \text{*}) (Integers modulo $33$ excluding $0$ under multiplication ) a group ? Motivate your answer.
\end{itemize}
asdasd

\begin{itemize}
    \item Write down a definition of homomorphism. Show a homomorphism
    between ($Z, +$) and ($R^+$, \text{*}).
\end{itemize}
asdasdasd

\begin{itemize}
    \item Estonian boatswain is trying to sell corsair Alice a sailboat, however without revealing the age of the sailboat. 
    From several sailors Alice found that one year ago, its age was a multiple of $4$, in $3$ years its age will be a multiple of $7$, 
    and in $5$ years multiple of $11$. Help Alice to deduce the age of the sailboat. Do not skip calculations in this solution!!!.
\end{itemize}
asdasdasd

\begin{itemize}
    \item Calculate the order of $7$ in ($U^{\text{*}}_{15}, \text{*}$) (set of units).
\end{itemize}
asdasdasd

\begin{itemize}
    \item Calculate Euler’s totient function for number $55440$.
\end{itemize}
asdasdasd


\begin{itemize}
    \item Calculate and provide intermediate calculations $$(-14)^{42}~mod~1181$$ 
\end{itemize}
asdasdasd



\section*{Task 2 (7 points) - Probability theory} % Numbered section
\begin{itemize}
    \item The box contains $3$ white balls and $2$ black ones. One ball is drawn from the box, and then the second. Event $B$ - the 
    appearance of a white ball at the first draw. Event $A$ - the appearance of a white ball on the second draw. Find $P(A|B)$ and $P(A|\overline{B})$
\end{itemize}
asdasdasd

\begin{itemize}
    \item Among the $30$ cryptography exam variants, there are $7$ ”lucky”. Two students take turns taking one variant 
    (the first student takes each of the variant with the same probability, the second - equally likely any of the remaining ones). 
    Find the probability that the second student took the ”lucky” variant. Write your answer as a fraction $\frac{X}{Y}$.
\end{itemize}
asdasdasd

\begin{itemize}
    \item Cryptography students is solving multiple choice task. There are n options. 
    If cryptography student knows a solution, they choose the right option. 
    If they do not - they randomly choose the answer with probability $\frac{1}{n}$. 
    What is the probability, that student knows a solution for a task, under a condition that they chose the right option? 
    (Hint: a) treat probability of student knowing the answer as parameter b) Bayes formula.
\end{itemize}
asdasdasd

\section*{Task 3 (8 points) - RSA} % Numbered section
\begin{itemize}
    \item Show that RSA is not ND-CCA2. The IND–CPA game is defined as follows.
\end{itemize}
\begin{figure}[H]
    \centering
    \label{fig:task3pic0}
    \includegraphics[scale=0.5, angle=0]{task3_pic0.png}
    \caption{Auction setup.}
\end{figure}

asdasd

\begin{itemize}
    \item Imagine the following modification to the $RSA$ encryption scheme. 
    Instead of generating $p$ and $q$ as a distinct prime numbers, we set $q = p$
    (i.e. modulus $n = p^2$). All other steps remain the same. Please, explain, 
    is the security of the scheme affected by this change? If yes, please provide an attack.
\end{itemize}

asd

\section*{Task 4 (6 points) - ElGamal encryption} % Numbered section
\begin{enumerate}
    \item Consider the ElGamal cryptosystems with a public key $(p,q,y)$ and a private key $x$. Encrypt the message $m = 15131$ 
    using parameters $p = 199999$,~$q = 23793$,~$x = 894$,~$r = 723$. Decrypt the ciphertext $c = (299,457)$ using parameters $p = 503$, $q = 2$, $x = 42$.
\end{enumerate}

\begin{enumerate}
    \setcounter{enumi}{1}
    \item Assume there is an jewellery auction happening, both Alice and Eve want to buy precious diamond earrings. 
    The rules of the auction are the following:
    \begin{itemize}
        \item Each bidder places a bid.
        \item The highest bidder gets the first slot, the second-highest, the second slot and so on.
        \item The highest bidder pays the price bid by the second-highest bidder.
    \end{itemize}
\end{enumerate}
\begin{figure}[H]
    \centering
    \label{fig:task4pic0}
    \includegraphics[scale=0.5, angle=0]{task4_pic0.png}
    \caption{Auction setup.}
\end{figure}

Eve found out that the bids are sent to the auction house in encrypted form, using ElGamal encryption. Additionally, 
Eve can eavesdrop on communication between Alice and auction house. Explain, how can Eve win the auction using homomorphic properties of ElGamal.

\section*{Task 5 (9 points) - Hash functions and signatures} % Numbered section
\textbf{Part 1.} Let p be a prime number and let $g$ be a generator of $Z^{\text{∗}}_{p}$. Suppose we have the following 
function: $f : Z \rightarrow Z^{\text{*}}_{p}$, where $f(x) = g^{x}~mod~p$. Is $f$ collision resistant? Please, explain your answer.
\\ \\
asd
\\ \\
\textbf{Part 2.} Assume Alice uses some signature algorithm $S$ with a hash function $H$ that is not collision resistant. 
That is, to sign a message $m$ Alice first computes $H(m)$ and then uses some secret key to sign $H(m)$. 
Explain, how malicious Carl can trick Alice into signing a fraudulent document.
\\ \\ 
asd


\end{document}
